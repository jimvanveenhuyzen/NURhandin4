\section*{Code of Problem 2}

The code is included below. Like in the previous hand-ins, I use code between sub-questions so I found that having one big .py file per problem is the most efficient.
\lstinputlisting{NURhandin4_2.py}

\newpage

\section*{Problem 2a}

The 2D slices of the density contrast $\delta$ are plotted below at $z$ = 4.5, 9.5, 11.5 and 14.5 respectively as a colormap. Note here that positive values in the colorbar on the side correspond to overdensities relative to the mean density of the volume, while negative values indicate underdensities.

\begin{figure}[h!]
  \centering
  \includegraphics[width=0.5\linewidth]{./2a_delta45.png}
  \caption{Plot of a 2D grid slice at z = 4.5 of the density contrast $\delta$. Positive/negative values correspond to an overdensity/underdensity compared to the mean density of the grid.}
\end{figure}

\begin{figure}[h!]
  \centering
  \includegraphics[width=0.5\linewidth]{./2a_delta95.png}
  \caption{Plot of a 2D grid slice at z = 9.5 of the density contrast $\delta$. Positive/negative values correspond to an overdensity/underdensity compared to the mean density of the grid.}
\end{figure}

\begin{figure}[h!]
  \centering
  \includegraphics[width=0.5\linewidth]{./2a_delta115.png}
  \caption{Plot of a 2D grid slice at z = 11.5 of the density contrast $\delta$. Positive/negative values correspond to an overdensity/underdensity compared to the mean density of the grid.}
\end{figure}

\begin{figure}[h!]
  \centering
  \includegraphics[width=0.5\linewidth]{./2a_delta145.png}
  \caption{Plot of a 2D grid slice at z = 14.5 of the density contrast $\delta$. Positive/negative values correspond to an overdensity/underdensity compared to the mean density of the grid.}
\end{figure}

\newpage

\section*{Problem 2b}

In problem b) we are asked to find gravitational potential $\Phi$, which we require to compute the gravitational forces at each grid point. As said in the problem, we can find cleverly this by Fourier transforming the Poisson equation, rewriting the result into a transformed potential by dividing by $k^2$ and finally inverse Fourier transforming this. To be clear, the definition of the Fourier- and inverse Fourier transform between 3-dimensional position space $\vec{x}$ and k-space $\vec{k}$ are given by:

\begin{equation}
\hat{f}(\vec{k}) = \int_{-\infty}^{\infty} f(\vec{x}) e^{i \vec{k} \cdot \vec{x}}d\vec{x} 
\end{equation}

\begin{equation}
f(\vec{x}) = \int_{-\infty}^{\infty} \hat{f}(\vec{k}) e^{-i \vec{k} \cdot \vec{x}}d\vec{k} 
\end{equation}

The way we solve this problem numerically is by implementing both a FFT and inverse FFT method for the 3-dimensional grid we have. In essence, the 3D FFT is the same as the 1D FFT, but now we have to split the grid into 1D slices and FFT each slice, looping through the array $3N^2$ times for both transforms. The specific algorithm we use to implement the FFT is the Recursive version of the \textbf{Cooley-Tukey} algorithm. My implementation of the algorithm is slightly different from the lecture notes. I first check if $N>1$, after which I split the input array into 2 seperate arrays containing the even and odd elements respectively, which I call recursively with $N = N/2$. Next, we seperately transform both arrays using the formulas given in the slides containing the complex exponential term, looping from $k=0$ to $k = N/2$. After this process we concatenate the arrays into one final, Fourier transformed array. This algorithm only works if $N$ is a power of 2, but luckily we are dealing with a grid of dimenions $16 \times 16 \times 16$, so this holds. Another relevant note is that the exponent we use in the algorithm is somewhat different compared to the one in the equation we just defined above. The exponent we use is given by $e^{i 2\pi k/N}$.\\

The inverse FFT is done exactly the same using Cooley-Tukey, but now we recursively call the inverse FFT instead and use the complex conjugate in the exponential term. Another important note is that we have to divide the result of the inverse FFT by $N$ to normalise the function properly. We do this outside of the function, because if we try to code this inside the function, it was recursively do this normalisation step and thus lead to the wrong result. Note that this whole situation with different exponents and normalisations is heavily dependent on definition, so there are  more ways to do a valid Fourier transform.\\

Moving on to the problem, we can compute the initial value of the Poisson equation $\nabla^2 \Phi = 4\pi G \bar{\rho} (1 + \delta)$, where $\bar{\rho}$ is the mean density and $\delta$ is a $16 \times 16 \times 16$ three dimensional grid of density contrast values $\delta_{i,j,k}$. We first apply the FFT on this total expression (just a rescaled density contrast) from which we find $k^2 \hat{\Phi}$, the product of the wave vector $\vec{k} = 2\pi \vec{n}/\lambda$ squared and the gravitational potential in k-space, $\hat{\Phi}(k)$. Here, $\vec{n}$ is the position space vector representing the grid, so it is again a $16 \times 16 \times 16$ vector, ranging from 0.5 to 15.5 along each axis. To further clarify this, the Fourier transformed grid point at $x =y = z = 0$ is divided by $0.5^2 + 0.5^2 +0.5^2$, and the one at (8,11,3) by $8.5^2 + 11.5^2 +3.5^2$. We use this in our calculating to find $k^2$, using the formula below:\\

\begin{equation}
k^2 = k_x^2 + k_y^2 + k_z^2 = (n_x^2 + n_y^2 + n_z^2) \Big(\frac{2\pi}{\lambda}\Big)^2
\end{equation}

The periodic boundary conditions of the volume are such that $x$ = 16 $\equiv$ 0, so the 'wavelength' of one period is 16 also. To convert this to just the transformed potential, we divide the result of the FFT by this $k^2$. As a final step we inverse FFT $\hat{\Phi}(k)$, which is proportional to $\hat{\delta}/k^2$. From this we obtain the final transformed potential, $\Phi$. This potential is plotted again below, which ends up as a smoothed out tracer of the density contrast.\\


\begin{figure}[h!]
  \centering
  \includegraphics[width=0.5\linewidth]{./2b_pot45.png}
  \caption{Plot of a 2D grid slice at z = 4.5 of the gravitational potential $\Phi$.}
\end{figure}

\begin{figure}[h!]
  \centering
  \includegraphics[width=0.5\linewidth]{./2b_pot95.png}
  \caption{Plot of a 2D grid slice at z = 9.5 of the gravitational potential $\Phi$.}
\end{figure}

\begin{figure}[h!]
  \centering
  \includegraphics[width=0.5\linewidth]{./2b_pot115.png}
  \caption{Plot of a 2D grid slice at z = 11.5 of the gravitational potential $\Phi$.}
\end{figure}

\begin{figure}[h!]
  \centering
  \includegraphics[width=0.5\linewidth]{./2b_pot145.png}
  \caption{Plot of a 2D grid slice at z = 14.5 of the gravitational potential $\Phi$.}
\end{figure}

\clearpage

The potential $\Phi$ appears to be a 'smoothing' of the overdensity values if we compare the plots with those from problem 2a, where the density values seem to have a rather random and chaotic distribution. This result is makes sense as you'd expect the gravitational potential and thus force to be a smooth function of the density $\rho$ (or $\delta$). Our knowledge from the study of large scale structure confirms this: overdensities are essentially just potential wells: regions of large potential energy, varying with the degree of overdensity. Next, we create the same plots, but this time for log$_{10}(|\Phi|)$, where we take the absolute value of the potential, most importantly because negative logarithms are undefined, so this prevents singularities! The plots are shown below. Note that this time, the colorbar at the side is logarithmic, so e.g. -8.7 refers to $10^{-8.7}$.\\

\begin{figure}[h!]
  \centering
  \includegraphics[width=0.5\linewidth]{./2b_log45.png}
  \caption{Plot of a 2D grid slice at z = 4.5 of the log$_{10}$ of the gravitational potential, log$_{10}(|\Phi|)$.}
\end{figure}

\begin{figure}[h!]
  \centering
  \includegraphics[width=0.5\linewidth]{./2b_log95.png}
  \caption{Plot of a 2D grid slice at z = 9.5 of the log$_{10}$ of the gravitational potential, log$_{10}(|\Phi|)$.}
\end{figure}

\begin{figure}[h!]
  \centering
  \includegraphics[width=0.5\linewidth]{./2b_log115.png}
  \caption{Plot of a 2D grid slice at z = 11.5 of the log$_{10}$ of the gravitational potential, log$_{10}(|\Phi|)$.}
\end{figure}

\begin{figure}[h!]
  \centering
  \includegraphics[width=0.5\linewidth]{./2b_log145.png}
  \caption{Plot of a 2D grid slice at z = 14.5 of the log$_{10}$ of the gravitational potential, log$_{10}(|\Phi|)$.}
\end{figure}

\clearpage












